\documentclass[11pt]{article}

\usepackage[utf8]{inputenc}
\usepackage[T1]{fontenc}
\usepackage[francais]{babel}
\usepackage[top=1.8cm, bottom=1.8cm, left=1.8cm, right=1.8cm]{geometry}
\usepackage{hyperref}
\hypersetup{
    colorlinks=true,
    breaklinks=true,
    urlcolor=red,
}
\parskip=5pt

\title{Rapport du projet de Fractale}
\author{Pierre AYOUB, Claire BASKEVITCH}
\date\today

\begin{document}

\maketitle

\section{Introduction}

Ce projet a pour but la modélisation de deux ensembles de fractales : l'ensemble
de Mandelbrot et l'ensemble de Julia et Fatou. Nous allons expliquer dans ce
rapport notre approche de ce projet et les choix que nous avons faits. Ce
document est voulu relativement abstrait du code, cependant, une description
détaillée des classes, méthodes et attributs est disponible. Pour cela, il
suffit de se rendre dans la documentation qui peut être générée avec Doxygen et
\LaTeX à partir des headers des classes.

\section{Fonctionnement}

Cette partie sera consacrée au fonctionnement du projet et à la manière dont
nous avons résolu certains problèmes, afin de répondre au cahier des charges.

\subsection{Choix architecturaux}

Après lecture et analyse du sujet, quatre choix majeurs s'offraient à nous :
GTK+ et Qt pour l'interface graphique, OpenGL et Cairo pour le rendu des
fractales.

De fil en aiguille, nos lectures nous ont fait conclure sur
l'utilisation de Qt pour l'interface graphique, en lieu et place de GTK+ :
\begin{itemize}
    \item Étant plus récent, Qt nous semblait plus actif en terme de
        développement et de documentation sur internet.
    \item Ce framework nous avait aussi l'air plus facilement portable entre les
        différents systèmes d'exploitation que GTK+.
    \item Qt, nativement développé en C++ et pensé pour l'orienté objet, nous
        paraissait plus simple à utiliser et adapté à un développement orienté
        objet.
    \item Enfin, Qt ne possède pas que des facultés d'interface graphique, car
        il permet de gérer des protocoles réseaux, d'interagir avec des bases de
        données, et autre. Pour cela, il nous paraissait intéressant de nous
        former avec Qt.
\end{itemize}

\vspace{10pt}
Concernant la bibliothèque de rendu à utiliser pour afficher nos fractales, nous
voulions faire une application qui proposerait le choix entre les deux.
Nous avons decidé de commencer à utiliser OpenGL car cette API nous plaisait
pour plusieurs raisons :
\begin{itemize}
    \item Curiosité de comprendre les bases du fonctionnement d'un jeu en 3D.
    \item API bas niveau, proche de la machine, offrant la possibilité de
        programmer directement pour le GPU (processeur graphique).
    \item Beaucoup d'extensions, ainsi que des possibilités d'utiliser OpenGL ES
        pour les systèmes embarqués, et surtout OpenCL pour le GPGPU qui
        pourrait nous servir plus tard.
\end{itemize}

\subsection{Qt}

L'interface du programme se présente sous une fenêtre principale contenant deux
onglets principaux, un pour chacun des ensembles. L'ensemble du code à été
réalise à la main sans outils supplémentaires, tel que QtDesigner pour ne citer
que lui, de manière à posséder une maîtrise total sur notre travail.
L'utilisateur peut afficher plusieurs fractales en même temps, nous avons jugé
cette fonctionnalité intéressante pour comparer les fractales entre elles. Les
différents widgets des de notre fenêtre sont positionnés grâce à des layouts, ce
qui permet d'avoir un code très flexible quant au rajout d'éléments, ou de leur
suppression. La gestion des signaux et des slots permet l'appel des fonctions
liées à l'affichage des fractales en fonction des paramètres choisis par
l'utilisateur.

L'utilisation de Qt s'est montré assez laborieuse au début, notamment au moment
de la mise en route du projet. En effet, il existe un grand nombre de sous
modules différents (Core, Widget, OpenGL, etc.) et un système de génération de
code intermédiaire (MOC) à mettre en place à la compilation. Une fois ce système
mis en place, Qt s'est avéré assez plaisant à utiliser, avec une documentation
bien rédigé.

Nous avons essayer d'organiser les classes de la manière la plus modulable
possible, en isolant les parties suivantes : la fenêtre principale, une
abstraction d'une fenêtre affichant une fractale, et une spécialisation de cette
classe pour OpenGL et Cairo. L'abstraction d'une fenêtre affichant une fractale
contient l'ensemble des propriétés concernant une fractale, ainsi que toutes les
fonctions générales à implémenter (zoom, déplacement, etc). Le design de cette
classe abstraite à été fastidieux à mettre en place car nous voulions utiliser
l'héritage multiple, mais la façon dont l'héritage est conçu dans Qt ne nous
permettais pas de le faire dans notre cas précis, même avec l'héritage virtuel.
Nous avons donc du repenser notre classe pour pallier à ce problème.

Nous nous arrêterons ici pour Qt, cette partie pourrait être plus longue et
détaillée, mais au vue de ce qu'il y a à dire sur OpenGL, nous préférons lui
garder une plus grande place dans ce rapport.

\subsection{OpenGL}

Sans surprise, OpenGL à été plus compliqué à aborder que Qt. Au fil de nos
lectures, il nous fallait comprendre plusieurs notions pour l'utiliser. Tout
d'abord, il nous fallait savoir qu'OpenGL était une API, standardisée par
Khronos, mais ne concernait pas une implémentation particulière. En effet, les
implémentation d'OpenGL sont propriétaires et conçu par les constructeurs de GPU
(Intel, AMD, Nvidia). Il existe aussi des implémentations libres et open source,
tel que Mesa. Ensuite, il existe aujourd'hui deux écoles concernant OpenGL :
l'OpenGL ancien (version $\leq$ 2), et l'OpenGL dit moderne (version $\geq$ 3).
L'OpenGL ancien serait plus simple à aborder pour des débutants, moins complexe
et automatisant beaucoup certaines tâches, cependant, ce n'est plus une approche
viable aujourd'hui pour un développement sérieux. L'OpenGL moderne est, quant à
lui, plus compliqué à apprendre pour des novices car il fait rentré en jeu plus
de notion dès le début, et automatise moins de tâches. Cependant, cet OpenGL
permet un contrôle élevé sur le pipeline 3D et les shaders, avec un accès plus
direct au matériel (encore plus poussé par Vulkan et DirectX 12).

Nous avons donc fait le pari d'apprendre l'OpenGL moderne pour nous familiariser
au mieux à OpenGL, dans l'optique qu'un jour nous pourrions l'utiliser de
nouveau. Notre rendu utilise donc les shaders plutôt que les fonctions proposées
par le lien internet du sujet, ces dernières étant dépréciées du fait qu'elles
font parties l'OpenGL ancien. 

Mais revenons rapidement sur le fonctionnement primaire d'OpenGL, pour bien
comprendre la suite du rapport. OpenGL permet de crée deux types de monde, un
monde en 2D ou un monde en 3D. Dans tous les cas, les objets sont vectoriels,
cela signifie que l'on caractérise les sommets des objets par des points dans un
monde virtuel plutôt que des pixels sur l'écran. Ce sera à l'étape de la
rastérisation, ou encore matricialisation, que l'on passera d'une système de
coordonnées virtuel à des pixels sur l'écran. Les mathématiques et les matrices
occupes une très grande place dans OpenGL. En effet, chaque opération sur les
objets correspond en réalité à une opération entre des matrices ou des vecteurs.
Il existe 3 matrices principalement utilisées dans un monde en 3D, que nous
allons présenter brièvement. La première est la matrice de modèle, qui permet de
passer des coordonnées d'un objet dans son propre référentiel à des coordonnées
correspondant au monde virtuel en 3D dans lequel nous évoluons. La seconde est
la matrice de vue, qui nous sert à passer des coordonnées d'un objet dans un
monde en 3D aux coordonnées d'un objet vue par une caméra placé un certain
point. Cette matrice permet en effet de simuler l'utilisation d'une caméra qui
se déplacerait sur les trois axes d'une scène. Enfin, la matrice de projection
permet de passer des coordonnées des objets vues par la caméra aux coordonnées
des objets à afficher sur un écran en 2D. Ces matrices étant complémentaires, il
n'est pas rare de croisé une matrice "Model-View-Projection" (MVP), qui est une
matrice résultant de la multiplication de ces trois matrices.

Dans notre application, toutes ces étapes sont gérés par notre programme en C++,
c'est-à-dire par notre CPU. C'est ici que les shaders, qui sont des programmes
s'exécutant sur le GPU, interviennent. Les shaders sont écris en GLSL, le
langage de shader lié à OpenGL. En effet, les GPU n'ayant pas les mêmes jeu
d'instruction que les CPU classique ont des langages dédiés différents des
langages de programmation classique. Le pipeline 3D correspond à l'ensemble des
étapes s'exécutant sur le GPU, et les shaders viennent s'y inscrire à plusieurs
étapes. Il existe une multitude d'étape dans le pipeline 3D, mais nous en
retiendrons 3 principales : le traitement par le shader de vertex, la
rastérisation (nous en avons parlé plus haut), et le traitement par le shader de
fragment. Le shader de vertex prend en entrée un vertex et ses coordonnées dans
l'espace, applique des transformations dessus, et envoie le vecteur résultant au
processus de rastérisation. Une fois fait, le résultat de la rastérisation est
envoyé au shader de fragment, qui traite des pixels ou des lots de pixel (défini
par OpenGL) à afficher sur l'écran. Ce traitement correspondant notamment à
l'application d'une couleur ou d'une texture sur le ou les pixels.

Maintenant que nous avons clarifier le rôle des matrices et des shaders dans
OpenGL, nous allons pouvoir parler de notre implémentation. Cette dernière
consiste à définir un objet en deux dimensions, d'une forme d'un rectangle, qui
est censé remplir tout l'écran quoi qu'il arrive (peut importe le rapport défini
par la longueur divisé par la hauteur de la zone d'affichage). Cet objet est
défini dans son propre référentiel, mais comme il doit au final être placé au
centre du monde, il n'y à pas besoin d'utiliser de matrice de modèle (elle sera
égale à une matrice identité). Nous passons ensuite directement cet objet et
tout les paramètres concernant la fractale à nos shaders. Le premier shader, le
shader de vertex, ne fait que multiplier chaque sommets (vertex) par la matrice
MVP, pour les projeter ensuite correctement à l'écran. Nous avons implémentés
les calculs concernant les itérations dans le shader de fragment, étant donné
qu'il y à un très grand nombre de calcul à faire pour chaque pixel et que le GPU
est hautement parallélisé, nous pensons que leurs place se trouvait dans ce
shader. Le shader de fragment exécute donc toutes les itérations de l'équation
à effectuer sur une unité de traitement du GPU, et détermine en fonction du
résultat de ces itérations la couleur du fragment (pour rappel, un pixel ou un
groupe de pixels). C'est au niveau de ce shader que l'on applique aussi le
facteur de zoom, ainsi que le déplacement dans l'image, mais nous y reviendrons.
Enfin, concernant les équations à implémenter, nous avons effectué une
décomposition des nombres complexes en deux réels stocker dans une structure de
point (x et y). Une fois décomposé ainsi, il suffisait de faire les
développements (identités remarquables) puis les réductions et factorisations
nécessaires sur nos calculs de nombre complexe pour savoir comment
l'implémenter. 

Nous allons maintenant pouvoir parler plus en détail du choix d'implémentation
du zoom et du déplacement dans la fractale. Nous en avons testé deux :
\begin{itemize}
    \item La première, consiste à transmettre au shader de fragment un facteur
        de zoom et un facteur de déplacement dans l'image. Ce shader s'occupe
        alors d'appliquer le facteur de zoom sur chaque fragment, ce qui aura
        pour effet d'augmenter ou diminuer leur taille dans notre objet sans
        modifier la taille de l'objet. De même, il applique le facteur de
        déplacement en translatant les fragments de manière à en faire sortir de
        la zone d'affichage de l'objet. Appliquer un zoom puis un facteur de
        déplacement revient donc à rogner l'afficher de notre objet, pour
        n'afficher que le centre que nous voulons afficher et ainsi avoir un
        zoom au sein même d'un objet dans l'espace en 3D.
    \item La seconde consiste à toujours garder notre objet intact (pas de
        rognage par agrandissement des fragments) et à zoomer par
        l'intermédiaire d'une caméra dans le monde en 3D (pour zoomer sur un
        objet en 2D). Cette caméra, implémentée avec une matrice de vue, permet
        un zoom et un déplacement pour l'utilisateur en faisant varier deux
        paramètres : la distance entre la caméra et l'objet, et le FOV (angle de
        champ, à la manière de la longueur focale d'un objectif d'appareil
        photographique).
\end{itemize}
Finalement, nous avons préféré garder la première méthode. En effet, une fois
les deux implémentées, la première permettait d'avoir une meilleur précision
d'affichage. Avec la deuxième méthode, des déformations apparaissait au niveau
des fragments lorsque l'on arrivait à 90\% du zoom, dû à des erreurs d'arrondis.
Cependant, elle avait pour avantage d'être légèrement plus rapide, mais nous
avons préféré privilégié la qualité graphique.

\subsection{Cairo}

Notre projet ne permet pas d'afficher des fractales avec Cairo. En effet, Qt et
OpenGL nous auront pris plus de temps que ce que nous imaginions. Au moment de
nous lancer dans Cairo, nous avons préféré peaufiner notre application
(meilleur abstraction des classes Qt, optimisation du code OpenGL, etc...)
plutôt que commencer une fonctionnalité qui n'aurait pas été terminé. Cependant,
voilà comment nous aurions pu faire : il aurait fallu crée une classe héritant
de FractalWindow nommée FractalWindowCairo. Tout comme FractalWindowOGL, il
aurait fallu mettre en place OpenGL avec le strict minimum. Mais au lieu
d'afficher notre fractale rendu par les primitives d'OpenGL, il aurait fallu
rendre une image en utilisant Cairo, stocker cette image en mémoire, puis
l'afficher avec OpenGL en utilisant la fonction << glTexImage2D >>. En effet, Qt
permet d'afficher un espace de rendu pour OpenGL, mais pas pour Cairo. La
solution était donc de charger le rendu de Cairo dans le contexte de rendu
OpenGL, lui-même à l'intérieur d'un widget Qt.

\subsection{Moteur de production}

Au commencement du projet, nous avions mis en place le build avec GNU make.
Étants habitués de cet outil qui nous convenait jusqu'à présent, nous n'avons
pas voulu nous encombrer par l'apprentissage d'un nouveau logiciel. Cependant,
GNU make sans complément (comme les GNU autotools) n'est pas très adapté pour
des projet utilisant de grosses bibliothèques, avec une multitude de versions et
d'emplacement possible sur le disque. C'est pour cela que la mise en route avec
Qt avait été un peu laborieuse, comme expliqué dans le paragraphe dédié plus
haut. Comme il était spécifié dans le sujet que nous devions crée un mécanisme
d'installation, nous avons décidé au milieu du projet d'apprendre à d'utiliser
CMake. Nous n'allons pas nous éterniser là-dessus ici, mais CMake propose de
générer des fichiers de build indépendant de la plateforme (Makefile pour UNIX,
Visual Studio pour Windows, etc.). De plus, il permet de rechercher
automatiquement sur le disque la présence de certains logiciel ou bibliothèques.
Cela permet donc d'avoir une compilation avec moins de risque de dépendances non
trouvées, et multi-plateforme. Les commandes à utilisées sont spécifiées dans le
<< README.md >> présent à la racine du projet. De plus, nos anciens Makefile
crées manuellement sont conservés dans un dossier << config\_old >>, mais ne
sont plus utilisés.

\section{Regard final}

Cette partie sera consacrée à notre ressenti après le projet, ce que l'on en
pense et en conclus.

\subsection{Critique sur le code et améliorations possibles}

Comme un code n'est jamais parfait et qu'il y à toujours quelques chose à
améliorer, le notre n'échappe pas à la règle. Voici plusieurs piste
d'amélioration.
\begin{itemize}
    \item \href{https://fr.wikipedia.org/wiki/God_object}{God Object} : nous
        nous demandons si notre classe MainWindow ne tombe pas dans cet
        anti-pattern. En effet, tout est relié à cette classe et passe par elle.
        De plus, nous n'avons rien d'autre dans notre << main >> que
        l'instanciation de cette classe. Cela ne pose pas de soucis pour
        l'implémentation du projet, mais ce serais intéressant de savoir si
        notre cas correspond à cette mauvaise pratique, pour ne pas retomber
        dans ce piège de l'orienté objet.
    \item Exception : notre projet ne présent que très peu de gestion d'erreur,
        ce qui est fort regrettable. Malheureusement, il nous aurais fallu plus
        de temps pour améliorer le design de notre application à ce niveau là.
    \item Mémoire : notre gestion de la mémoire est un peu aléatoire. En effet,
        nous procédons à beaucoup d'allocation dynamique pour deux raisons : Qt
        travaille avec des pointeurs, et cela nous permet de mieux contrôler la
        durée de vie de nos objets (en théorie). Cependant, d'après nos lecture
        (\href{https://doc.qt.io/}{documentation officiel de Qt} ou forums de
        programmation tel que \href{https://stackoverflow.com/}{Stack
        Overflow}), il nous est assuré que Qt s'occupe lui-même de libérer la
        mémoire des objets héritants de QWidget. Cela lui est possible en
        gardant constamment une hiérarchie en mémoire des widgets instanciés
        grâce à leurs liens de parenté spécifiés lors de l'instanciation du
        widget ou par l'agencement des layouts. Cependant, nous avons observer
        des comportements étranges au niveau de la mémoire relatif à notre
        application. Malheureusement, nous n'avons pas été capable d'observer
        précisément ce qu'il se passait avec Valgrind car Qt et OpenGL
        produisent beaucoup d'erreurs de base, qui sont des faux positifs par
        rapport a notre programme car ces erreurs ne dépendent pas de nous. Sans
        temps supplémentaire pour analyser cela en profondeur, nous avons du
        laisser cela en l'état.
\end{itemize}

\section{Conclusion}

Nous sommes un peu mitigés sur notre travail. D'une part, nous sommes
globalement satisfaits de notre projet sur Qt et OpenGL, où nous avons essayé
d'approfondir au mieux nos connaissances afin de produire un code clair et
efficace. D'autre part, par manque de temps, nous n'avons pas pu réaliser le
rendu graphique avec Cairo, ce qui peut être ressentit comme un échec malgré la
volonté d'aller plus loin.

Ce projet nous aura tout de même permis d'avoir une première vraie expérience du
C++, qui manquait à notre bagage de développeurs. Nous nous sommes rendu compte
que c'est un langage tellement vaste que nous n'auront jamais fini d'en
apprendre ! Nous auront pu approfondir nos connaissances sur l'héritage,
l'héritage virtuel, les fonctions virtuels, la liaison tardive, et d'autre
notion du C++ que nous avons vu en cours.

De plus, l'apprentissage de Qt et OpenGL aura aussi été très bénéfique pour
nous. D'un côté, nous avons un framework ultra-complet qui nous permet de
développer rapidement des applications multi-platerforme avec tout un lot de
fonctionnalités. De l'autre côté, nous avons les connaissances de bases du
développement d'un jeu vidéo ou d'un logiciel en 3D, et éventuellement des bases
qui pourrait nous permettre aussi d'aborder plus tard le GPGPU et le calcul
scientifique avec OpenCL (puisque nous sommes tout les deux intéressé entre
autre par le master en calcul haute performance et simulation). Nous sommes
persuadés que nous auront à réutiliser tout ça plus tard, donc nous le voyons
comme un apprentissage important. 

Enfin, nous avons pu pratiquer de manière optimisée l'utilisation de GDB, avoir
un moteur de production multi-plateforme hautement personnalisable (CMake),
approfondir nos connaissances sur bon nombre de fonctionnalités du formidable
logiciel de contrôle de version Git. De surcroît, nous avons revu les bases du
langage LaTeX. Merci pour ce projet.

\end{document}
