\documentclass[11pt]{article}

\usepackage[utf8]{inputenc}
\usepackage[T1]{fontenc}
\usepackage[francais]{babel}
\usepackage[top=1.8cm, bottom=1.8cm, left=1.8cm, right=1.8cm]{geometry}
\parskip=5pt

\title{Rapport du projet de Fractale}
\author{Pierre AYOUB, Claire BASKEVITCH}
\date\today

\begin{document}

\maketitle

\section{Introduction}

Ce projet a pour but la modélisation de deux ensembles de fractales : 
l'ensemble de Mandelbrot et l'ensemble de Julia et Fatou.
Nous allons expliquer dans ce rapport notre approche de ce projet 
et les choix que nous avons faits.
Une description détaillées des classes, méthodes et attributs est 
disponible dans la documentation.

\section{Fonctionnement}

Cette partie sera consacrée au fonctionnement interne du projet
et à la manière dont nous avons résolu certains problèmes, afin de 
répondre au cahier des charges. Le projet est organisé comme suit : 
l'exécutable sera produit dans le dossier "exe", les sources dans le dossier "src"
et les header dans le dossier "inc".
Les headers contiennent les définitions des classes et la description du rôle
de chaque fonction (documentation). Les sources contiennent le code source
et les commentaires nécessaires à la compréhension du fonctionnement
interne de la fonction.

\subsection{Description des classes}

Les classes utilisées sont définies dans les fichiers ".hpp". Les headers 
sont assez documentés pour connaître le rôle de tous les attributs et 
méthodes des classes, nous ne nous y attarderons donc pas ici.


paragraphe sur Qt.


Concernant le développement avec OpenGL, nous avons décidé d'utiliser 
les shaders plutôt que les fonctions proposées via le lien internet du 
sujet. En effet, ces dernières étant dépréciées, nous avons pensez 
qu'il était préférable d'apprendre à utiliser les fonctions actuelles 
d'OpenGL pour, d'une part, un rendu plus rapide de la modélisation et 
d'autre part nous familiariser au mieux avec OpenGL dans l'optique 
qu'un jour nous pourrions l'utiliser de nouveau. 

\subsection{Description des algorithmes}

Paragraphe.

\section{Regard final}

(note pour Pierre : je sais pas s'il vaut mieux mettre cela ici ou ailleurs
genre dans la conclusion)
Nous sommes mitigés sur notre travail. D'une part nous sommes globalement 
satifaits de notre travail sur Qt et OpenGL où nous avons essayé 
d'approfondir au mieux nos connaissances sur ces deux librairies afin de 
produire un code clair et efficace. D'autre part, par manque de temps, 
nous n'avons pas pu réaliser le rendu graphique avec CAIRO.

\subsection{Critique sur le code et améliorations possibles}

Paragraphe.

\subsection{Questions suite au projet}

Paragraphe.

\section{Conclusion}

Paragraphe.

\end{document}
